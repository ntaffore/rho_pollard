\documentclass[a4paper,10pt]{report}
\usepackage[utf8x]{inputenc}
\usepackage[french]{babel}
\usepackage[T1]{fontenc}
\usepackage{latexsym}
\usepackage{algorithm,algorithmic}


%%% francisation des algorithmes
\renewcommand{\algorithmicrequire} {\textbf{\textsc{Entrées:}}}
\renewcommand{\algorithmicensure}  {\textbf{\textsc{Sorties:}}}
\renewcommand{\algorithmicwhile}   {\textbf{tantque}}
\renewcommand{\algorithmicdo}      {\textbf{faire}}
\renewcommand{\algorithmicendwhile}{\textbf{fin tantque}}
\renewcommand{\algorithmicend}     {\textbf{fin}}
\renewcommand{\algorithmicif}      {\textbf{si}}
\renewcommand{\algorithmicendif}   {\textbf{finsi}}
\renewcommand{\algorithmicelse}    {\textbf{sinon}}
\renewcommand{\algorithmicthen}    {\textbf{alors}}
\renewcommand{\algorithmicfor}     {\textbf{pour}}
\renewcommand{\algorithmicforall}  {\textbf{pour tout}}
\renewcommand{\algorithmicdo}      {\textbf{faire}}
\renewcommand{\algorithmicendfor}  {\textbf{fin pour}}
\renewcommand{\algorithmicloop}    {\textbf{boucler}}
\renewcommand{\algorithmicendloop} {\textbf{fin boucle}}
\renewcommand{\algorithmicrepeat}  {\textbf{répéter}}
\renewcommand{\algorithmicuntil}   {\textbf{jusqu'à}}


% Title Page
\title{Algorithme rho de pollard}
\author{TAFFOREAU Nicolas}


% -----------------------------------------
% organisation a revoir 
% -----------------------------------------

\begin{document}
\maketitle
\tableofcontents

\chapter{Rapel sur les courbes elliptique}
\begin{description}
  \item[définition]

    Soit un un corp Fp (p premier) de caractéristique différente de 2 ou 3, soit a, b app Fp. Un courbe elliptique est définit par
    le point à l'infini et l'ensemble des points (x,y) tel que soit satifait l'equation suivante :
  \begin{center}
    $y^2 = x^3 + a*x + b$.   
  \end{center}
  Ces points forme un groupe abélien avec comme zero le point à l'infini et un loi de groupe.
  Soit P, Q app F, $ P = (x_1,y_1)$, $Q = (x_2,y_2)$ alors $P+Q = (x_3,y_3)$ tel que 
    \begin{center}
      $x_3 = \mu^2 -x_1 - x_2$, $y_3 = \mu(x_1-x_3)$
      \newline
      $\mu = \frac{y_2 - y_1}{x_2 - x_1}$ si $P <> Q$
      \newline
      $\mu = \frac{3*x_1^2 + a}{2*y_1}$ si $P = Q$
    \end{center}

\end{description}

propriété importante pour les inverse pour la loi d'addition sur Fp. Si $P = (x_1,y_1)$ alors $-P = (x_1,-y_1)$.
(sur Fq) ?
\chapter{Algorithme rho de Pollard}
L'algorithme rho de pollard est basé sur le paradoxe des anniversaire. Dans un groupe finit si on prend des éléments de façon aléatoire,
il faut en moyenne tirer $\sqrt{n}$ éléments pour obtenir un élément que l'on à déjà tiré, ou n est le nombre d'élément de notre ensemble.
Le nom de cette algorithme est dû à la forme que prend la suite aléatoire, on a d'abord un pré-cycle puis le cycle.
% \includegraphics{image de la lettre rho}
\section{dans un groupe du type Fp}
Dans un groupe du type Fp, Pollard a choisit comme fonction de tirage aléatoire un fonction qui ne tient compte que de l'élément précédent.\\
$ f(x) = x+P$ si $x \in [0;p/3[$\\
$ f(x) = x+x$ si $x \in [p/3;2p/3[$\\
$ f(x) = x+Q$ si $x \in [2p/3;p[$\\
Ici P est un generateur du groupe et Q est l'element dont on veut connaitre le logarithme en base P.
{
\begin{algorithm}
\caption{rho pollard}
\begin{algorithmic}
 \REQUIRE n,P,Q
 \ENSURE x tel que $Q = P^x mod n$
\end{algorithmic}
\end{algorithm}


\section{sur les courbe elliptique}
L'algorithme rho de pollard est identique sur les courbes elliptiques sauf que l'on se place dans un cas plus générale
avec en utilisant l'addition et des scalaires.

// ecriture de l'algo sur les courbe elliptique.

\section{algorithme de floyd}
Il existe différent moyen pour trouver une collision dans un cycle, le meilleur algorithme pour en trouver
une est l'agorithme de floyd qui permet de calculer de façon très rapide la longueur d'un cycle en se basant sur des collisions.
La propriété qu'il a énoncé est que dans un groupe cyclyque si on a une marche aléatoire cyclique, alors au lieu de stocké chaque 
éléments et de vérifier si il n'est pas dans la liste. il calcule l'élément i et 2i de cette marche aléatoire juqu'à avoir une 
collision.

exemple sur le groupe Z/17Z, avec la mache aléatoire $f(x) = x^x + 1 mod 17$.
si x = 1 -> 2 -> 5 -> 9 -> 13 -> 16 -> 0 -> 1
donc $a_1 = 1, a_2 = 2, a_3 = 5 .....$
// mauvais choix puissance de 2


\chapter{Negation map}
\section{methode general}
sur une courbe elliptique sur Fp l'inverse de P = (x,y,z), simplifié à (x,y) est -P = (x,-y). le but de cette méthode est de reduire le groupe de moitier,
donc de faire une recherche de log discret dans <P>/<H>. On obtient donc une relation du type $\pm[a]P \pm[b]Q = \pm[a']P \pm[b']Q$. Il nous suffit donc de retrouver exactement 
la relation avec les bon signe pour retrouver le logarithme discret.

premier probleme on retombe très rapidement sur le meme point avec les meme scalaires	

\section{cycle infructueux}
apppartition très fréquente de cycle de taille 2 n'important pas d'information sur le log discret on les elimine donc en ......

\chapter{nombre de groupe r}
declaration d'un fonction qui trie dans r groupe different.


\chapter{random walk}

Mon premier choix de marche aléatoire était de prendre les scalaires a et b puis de les regarder modulo 3 pour les séparer dans les 
trois groupe différent et donc faire l'oppération que d'ajout de P ou de Q, ou le doublement.
La première remarque avec les différents test que je faisait était que je mettait plus de temps à trouver le logarithme discret
qu'une recherche exaustive. J'ai donc remarquer que la plupart du temps la fonction qui prenait un point de la courbe comme argument,
ainsi que les scalaires a et b, ne retournait pas à chaque fois la même image pour un point donné. Concretement si 
$W1 = [2]P \oplus [3]Q$ et $W2 = [4]P \oplus [5]Q$ avec W1 = W2 alors il n'avait pas le même resulta par f.

point important : plus trop une marche alléatoire pour 3 groupes donc plus pour r groupes r > 19.

Pour le choix de la marche \textbf{Edlyn Teske} a écrit dans \textit{ON RANDOM WALKS FOR POLLARD’S RHO METHOD} un indice L  =  # nombre iteration avant la colision sur $sqrt(|G|)$.


\chapter{endomorphisme}

La fonction f décrivant une marche aléatoire si elle est pris de façon  à avoir f(W) = f(-W), alors d'après les test on a une réduction qui doit théoriquement d'après le papier de breinstein être de $sqrt(2)$.
Les test montre qu'on a plutôt une réduction de l'ordre de $????????????$.
Pour ce ramener à la méthode de la négation map qui divise la taille du groupe en 2, ici on fait plus ou moins la même chose car dans le negation map, on garde le min de y pour P  ou -P, ici on fait que la fonction 
f ait les même valeur pour P et -P.

\chapter{parrallelisation}
pour parralelisé le log discret on definie une règle pour avoir des point remarquable (tel que ....) 
a chaque fois que l'on a après une marche un point remarquable on le transmet a un serveur central ainsi que les donné de départ
quand on  a deux intersection de point remarquable on a une collision.

\chapter{bibliograpie}
\end{document}          
